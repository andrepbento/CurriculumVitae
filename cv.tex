\documentclass[a4paper,9pt]{memoir}

\usepackage{hyperref}
\usepackage{marvosym}
\usepackage{fontawesome}
\usepackage{lmodern}

\hypersetup{
    colorlinks=true,
    linkcolor=black,
    filecolor=black,
    urlcolor=black,
}

\usepackage{XCharter} % Use the Bitstream Charter font
\usepackage[utf8]{inputenc} % Required for inputting international characters
\usepackage[T1]{fontenc} % Output font encoding for international characters

\usepackage[top=1cm,left=1cm,right=1cm,bottom=1cm]{geometry} % Modify margins

\usepackage{graphicx} % Required for figures

\usepackage{url} % URLs

\usepackage[usenames,dvipsnames]{xcolor} % Required for custom colours

\usepackage{enumitem} % Required for modifying lists
\setlist{noitemsep,nolistsep} % Remove spacing within and around lists

\pagestyle{empty} % Disable all page numbering

\setlength{\parindent}{0pt} % Stop paragraph indentation

\newcommand{\userinformation}[1]{\renewcommand{\userinformation}{#1}} % Define a new command for the CV user's information that goes into the left column
\newcommand{\cvheading}[1]{{\huge\bfseries\color{RoyalBlue} #1}} % New command for the CV heading
\newcommand{\cvsubheading}[1]{{\large\bfseries #1} \bigbreak} % New command for the CV subheading
\newcommand{\Sep}{\vspace{1em}} % New command for the spacing between headings
\newcommand{\SmallSep}{\vspace{0.5em}} % New command for the spacing within headings
\newcommand{\aboutme}[2]{ % New command for the about me section
    \textbf{\color{RoyalBlue} #1}~~#2\par\Sep
}
\newcommand{\CVSection}[1]{ % New command for the headings within sections
    {\Large\textbf{#1}}\par
    \SmallSep % Used for spacing
}
\newcommand{\CVItem}[2]{ % New command for the item descriptions
    \textbf{\color{RoyalBlue} #1}\par
    #2
    \SmallSep % Used for spacing
}
\newcommand{\bluebullet}{\textcolor{RoyalBlue}{$\circ$}~~} % New command for the blue bullets

\begin{document}

\begin{center}
    \cvheading{\textbf{André Pascoal Bento}}

    \cvsubheading{\Large PhD Student @ CISUC | Assistant Professor @ University of Coimbra}
    %\noindent\rule{\linewidth}{0.4pt}
    
    \faEnvelope~\href{mailto:apbento@dei.uc.pt}{apbento@dei.uc.pt} \quad
    \faPhone~(+351)~910~349~466 \quad
    \textbf{Address:} Coimbra, Portugal.

    \faLinkedin~\href{https://www.linkedin.com/in/andre-bento/?locale=en_US}{andre-bento} \quad
    \faGlobe~\href{https://www.cisuc.uc.pt/en/people/apbento}{CISUC Profile} \quad
    \faGoogle~\href{https://scholar.google.com/citations?user=9Yl9gBwAAAAJ&hl=en}{Scholar} \quad
    \faGithub~\href{https://github.com/andrepbento}{andrepbento}

    \textbf{Communication skills:}
    Portuguese (native),
    English (advanced),
    French (intermediate),
    Spanish (intermediate).
    
    \noindent\rule{\linewidth}{0.4pt}
\end{center}


\aboutme{About}{
André Bento is a researcher at the Centre for Informatics and Systems at the University of Coimbra, where he is also pursuing a Ph.D. in Informatics Engineering and teaching as an invited professor. His research focuses on optimizing availability and resource utilization of cloud services.
%
He holds a B.Sc. from the Coimbra Institute of Engineering and a M.Sc. from the University of Coimbra. Passionate about distributed systems, he constantly seeks new learning opportunities. Beyond work, he enjoys swimming, cycling, exploring nature and playing the guitar.
}

\CVSection{Experience}

\CVItem{Sep 2021 - Present, \textit{Invited Professor}, University of Coimbra}{
    Teaching practical laboratory classes on Distributed Systems (B.Sc.) and Systems Integration (M.Sc.) in Informatics Engineering. Topics in Distributed Systems include multi-threading, parallel programming, RMI/RPC, REST services (Spring Boot and FastAPI), and WebSockets. Systems Integration topics cover microservices, service-oriented architectures, integration patterns, data serialization formats (XML, JSON, and Protocol Buffers), Reactive REST APIs and event-driven systems (Apache Kafka).
}

\CVItem{Sep 2019 - Present, \textit{Researcher}, CISUC - Centre for Informatics and Systems (University of Coimbra)}{
    Researching optimization techniques and root-cause analysis to improve availability and resource utilization of cloud services.
    Developing solutions using Docker, Kubernetes, Helm, Terraform, Ansible, CI/CD and AWS, alongside service mesh (Istio), observability tools (Grafana, Prometheus, and Jaeger) and data analysis using Pandas, NumPy, and SciPy. Utilizing Python, Java, and Go as programming languages.
}

\CVItem{Sep 2018 - Jul 2019, \textit{Research Intern}, CISUC - Centre for Informatics and Systems (University of Coimbra)}{
    Researched microservices, observability and performance monitoring using metrics, logs, and distributed tracing.
}

\CVItem{Feb 2017 - Jul 2017, \textit{Software Engineer Intern}, WIT Software, S.A.}{
    Developed a Mobile Augmented Reality prototype for Android and iOS, implementing digital filters, image manipulation, and user content creation features (\textit{e.g.}, selfies, stickers, photo effects, emojis, and drawings).

}

\CVItem{Oct 2016 - Jun 2017, \textit{Scratch Teacher Assistant}, CASPAE 10}{
    Taught problem-solving using Scratch programming to primary school children in 3rd and 4th grades.
}

\CVItem{Nov 2015 - Mar 2016, \textit{Math Applied to Engineer Teacher -- Volunteer}, CeAMatE}{
    Taught mathematics to pre-degree and engineering students.
}

\CVItem{May 2014 - Jul 2014, \textit{Accountant Technician Intern}, Caixa de Crédito Agrícola Mútuo de Mira - C.R.L.}{
    Assisted with bank accounting, organization, and daily operations.
}

\CVItem{May 2013 - Jun 2013, \textit{Accountant Technician Intern}, Caixa de Crédito Agrícola Mútuo de Cantanhede - C.R.L.}{
    Supported bank accounting and daily operational tasks.
}

\Sep

\CVSection{Education}

\CVItem{2019 - 2025 (Expected), Ph.D. in Informatics Engineering, University of Coimbra}{
\emph{Thesis: Optimizing Availability and Resource Utilization of Cloud Services.}
}

\CVItem{2017 - 2019, M.Sc. in Informatics Engineering, University of Coimbra}{\emph{Thesis: Observing and Controlling Performance in Microservices.}}

\CVItem{2014 - 2017, B.Sc. in Informatics Engineering, Coimbra Institute of Engineering (ISEC)}

\Sep

\CVSection{Grants and Projects}

\begin{itemize}[leftmargin=*]
    \item \textbf{Dec 2021 - Jul 2025, Ph.D. grant} - Foundation for Science and Technology (FCT) Grant number BD.06012.2021.
    \item \textbf{2023 - 2025, ROAR-NET} - Randomised Optimisation Algorithms Research Network (CA22137). Funded by COST.
    \item \textbf{2019 - 2022, AESOP} - Autonomic Service Operation. Funded by P2020-31/SI/2017, No. 040004.
\end{itemize}

\Sep

\CVSection{Publications}

\begin{enumerate}[leftmargin=*]
	\item \textbf{Andre Bento}, Filipe Araujo, Luís Paquete, and Raul Barbosa. 
	\emph{Optimal Scaling of Cloud Services}. Submitted to an International Journal.
	% \item \td{Andre Bento, Filipe Ribeiro, António Ferreira, Filipe Araujo, and Raul Barbosa. Modeling the Availability of Cloud Service Replicas. [WIP submitted to European Dependable Computing Conference [EDCC 2025]]}.
	\item \textbf{Andre Bento}, Filipe Araujo, and Raul Barbosa. \emph{Cost-availability aware scaling: Towards optimal scaling of cloud services}. Journal of Grid Computing, 21(4):80, 2023.% [Rank Q1 by Scimago Journal \& Country Rank].
	\item Gonçalo Baptista, Jaime Correia, \textbf{Andre Bento}, Joao Soares, Antonio Ferreira, Joao Duraes, Raul Barbosa, and Filipe Araujo. \emph{Defektor: An extensible tool for fault injection campaign management in microservice systems}. In Proceedings of the 38th ACM/SIGAPP Symposium on Applied Computing, SAC'23, page 184-187, New York, NY, USA, 2023.% Association for Computing Machinery. (Poster paper presented in conference at Tallinn, Estonia, March 27-31, 2023).% [Rank B by ERA]. 
	\item Stanley Lima, Filipe Araujo, Miguel de Oliveira Guerreiro, Jaime Correia, \textbf{Andre Bento}, and Raul Barbosa. \emph{Efficient causal access in geo-replicated storage systems}. Journal of Grid Computing, 21(1):8, 2023.% [Rank Q1 by Scimago Journal \& Country Rank].
	% \item Stanley Lima, Filipe Araujo, Miguel de Oliveira Guerreiro, Jaime Correia, \textbf{Andre Bento}, and Raul Barbosa. Optimal causal access in geo-replicated storage systems. 2022. Research Square - \td{Preprint}.
	\item \textbf{Andre Bento}, Joao Soares, António Ferreira, Joao Duraes, José Ferreira, Rita Carreira, Filipe Araujo, and Raul Barbosa. \emph{Bi-objective optimization of availability and cost for cloud services}. In 2022 IEEE 21st International Symposium on Network Computing and Applications (NCA), volume 21, pages 45-53. IEEE, 2022.% [Rank A by ERA].
	\item \textbf{Andre Bento}, Jaime Correia, Joao Duraes, João Soares, Luís Ribeiro, António Ferreira, Rita Carreira, Filipe Araujo, and Raul Barbosa. \emph{A layered framework for root cause diagnosis of microservices}. In 2021 IEEE 20th International Symposium on Network Computing and Applications (NCA), pages 1-8. IEEE, 2021.% [Rank A by ERA].
	\item João Tomás, \textbf{Andre Bento}, João Soares, Luís Ribeiro, António Ferreira, Rita Carreira, Filipe Araújo, and Raul Barbosa. \emph{Autonomic service operation for cloud applications: Safe actuation and risk management}. In Dependable Computing-EDCC 2021 Workshops: DREAMS, DSOGRI, SERENE 2021, Munich, Germany, September 13, 2021, Proceedings 17, pages 39-46. Springer, 2021.% [Rank B2 by Qualis].
	\item Sara Silva, Jaime Correia, \textbf{Andre Bento}, Filipe Araujo, and Raul Barbosa. \emph{$\mu$Viz: Visualization of microservices}. In 2021 25th 
	International Conference Information Visualisation (IV), pages 120-128. 
	IEEE, 2021.% [Rank B by ERA].
	\item \textbf{Andre Bento}, Jaime Correia, Ricardo Filipe, Filipe Araujo, and Jorge Cardoso. \emph{Automated analysis of distributed tracing: Challenges and research directions}. Journal of Grid Computing, 19:1-15, 2021.% [Rank Q1 by Scimago Journal \& Country Rank].
\end{enumerate}

\end{document}
